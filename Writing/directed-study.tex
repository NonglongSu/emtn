\documentclass[12pt,twoside]{article}
\usepackage{amsmath}

\begin{document}

\begin{abstract}
Models of molecular evolution are important for species conservation because they help us to determine whether two populations belong to the same species or different species. Thus, we can determine if they need to be preserved as one or as two distinct species.
The Tamura-Nei model is the most complex reversible model for molecular evolution for which transition probabilities can be explicitly calculated. Usually numerical optimization is used to estimare parameters for this model but instead we will use expectation-maximization (EM) to optimize parameters. They key for expectation-maximization is that it guarantees to find a local maximum and then can be added into other more complex models that already use EM. With EM we can then integrate the Tamura-Nei model into a larger context of models of conservation biology.
\end{abstract}

\section{Models of DNA Evolution and Phylogenetics}
Comparing genome sequences of different organisms from different or from the same species it can be seen that these change over time. These changes or mutations happen through their evolutionary history and can be produced by different causes. Sometimes mutations can produce a fixed polymorphism, which is the occupation of more than one allele at the same gene locus, and be transmitted to their descendants.
Phylogeny is the branch of biology that studies the evolutionary development of a species or a taxonomic group. Phylogenetics is focused on the history of a species through sequencing data. It shows the relationship, differences and similarities, among their evolutionary history. Phylogenetic studies are based on the comparison of genomes from different species allows to estimate the historic relationship between them and their distance in time.
Estimating the genetic distance between two homologous sequences is measuring the number of differences accumulated between them since they diverge from a common ancestor. It is non trivial to estimate this distance since multiple substitutions can happened, thus the phylogenetic analysis relays on choosing the appropriate substitution model, also called model of DNA evolution. Each model of DNA evolution sets a ratio of substitution per unit of time as well as the frequency of all four DNA bases.

\section{Tamura-Nei Model}
The Tamura-Nei model of DNA evolution contemplates the probability of having, at any site, two kind of events: transitions and transversions. If a site has a purine, it assumes that there is a constant probability of $\alpha_{R}$ per unit time of replacing the base with a random purine, first case of event of type I (transition). Similarly, if a site has a pyrimidine, the model assumes there is a constant probability of $\alpha_{Y}$ per unit of time of the base being replaced by a random pyrimidine, second case of event type I (transversion). The Tamura-Nei model also assumes that every site has a constant probability of $\beta$ per unit of time of the base being replaced by a base drawn from all four bases (overall pool), type II event. The overall pool is assumed to contain the four different bases at some frequencies $\pi_{A}$, $\pi_{C}$, $\pi_{G}$, $\pi_{T}$ and these will particularly be the equilibrium frequencies.

\section{Methods}

\section{Validation}
Validation of results was done using DNA Assembly with Gaps (Dawg) \cite{dawg} an application that simulates the evolution of recombinant DNA sequences.


\bibliographystyle{plain}
\bibliography{directed-study}

\end{document}
